  % CVPR 2022 Paper Template
  % based on the CVPR template provided by Ming-Ming Cheng (https://github.com/MCG-NKU/CVPR_Template)
  % modified and extended by Stefan Roth (stefan.roth@NOSPAMtu-darmstadt.de)

  \documentclass[10pt,twocolumn,letterpaper]{article}

  %%%%%%%%% PAPER TYPE  - PLEASE UPDATE FOR FINAL VERSION
  \usepackage[review]{cvpr}      % To produce the REVIEW version
  %\usepackage{cvpr}              % To produce the CAMERA-READY version
  %\usepackage[pagenumbers]{cvpr} % To force page numbers, e.g. for an arXiv version

  % Include other packages here, before hyperref.
  \usepackage{graphicx}
  \usepackage{amsmath}
  \usepackage{amssymb}
  \usepackage{booktabs}


  % It is strongly recommended to use hyperref, especially for the review version.
  % hyperref with option pagebackref eases the reviewers' job.
  % Please disable hyperref *only* if you encounter grave issues, e.g. with the
  % file validation for the camera-ready version.
  %
  % If you comment hyperref and then uncomment it, you should delete
  % ReviewTempalte.aux before re-running LaTeX.
  % (Or just hit 'q' on the first LaTeX run, let it finish, and you
  %  should be clear).
  \usepackage[pagebackref,breaklinks,colorlinks]{hyperref}


  % Support for easy cross-referencing
  \usepackage[capitalize]{cleveref}
  \crefname{section}{Sec.}{Secs.}
  \Crefname{section}{Section}{Sections}
  \Crefname{table}{Table}{Tables}
  \crefname{table}{Tab.}{Tabs.}


  \begin{document}

  %%%%%%%%% TITLE - PLEASE UPDATE
  \title{\LaTeX\ Author Guidelines for \confName~Proceedings}

  \author{First Author\\
  Institution1\\
  Institution1 address\\
  {\tt\small firstauthor@i1.org}
  % For a paper whose authors are all at the same institution,
  % omit the following lines up until the closing ``}''.
  % Additional authors and addresses can be added with ``\and'',
  % just like the second author.
  % To save space, use either the email address or home page, not both
  \and
  Second Author\\
  Institution2\\
  First line of institution2 address\\
  {\tt\small secondauthor@i2.org}
  }
  \maketitle


  %%%%%%%%% BODY TEXT
  \section{Introduction}
  \label{sec:intro}

Sign language is the most important way for the deaf/ people who are hard of hearing to communicate with each other. The problem faced is to identify the twenty-six characters of the English alphabet, the white space and the empty space. The most probable challenges that will be faced will be the correct classification of similar looking alphabets. By using Convolutional Neural Networks (CNN), we are trying to correctly classify the image as a character. For the purpose of diversifying the results we will be using the Mobile-Net v3, Shufflenet and NASNet-A-Mobile neural network to model three different datasets and then later train two models by making use of Transfer Learning.
Application: The conversion of sign language to text can prove to be quite helpful to not only the deaf people but also to others who do not understand the language. Furthermore, it will save costs of employing an intermediary for conversation between a person who uses sign language user and one who does not. It will also lead to better documentation, as saving long videos can be quite memory inefficient but saving the conversations in the form of text can be quite helpful.
Goal: The primary goal of the project is to predict the textual counterpart of a sign language image with high accuracy. In the end we will evaluate the accuracy using confusion matrix by comparing the results for each of the characters.


  %------------------------------------------------------------------------
  \section{Datasets Selection}
  \label{sec:formatting}

  All text must be in a two-column format.
  The total allowable size of the text area is $6\frac78$ inches (17.46 cm) wide by $8\frac78$ inches (22.54 cm) high.
  Columns are to be $3\frac14$ inches (8.25 cm) wide, with a $\frac{5}{16}$ inch (0.8 cm) space between them.
  The main title (on the first page) should begin 1 inch (2.54 cm) from the top edge of the page.
  The second and following pages should begin 1 inch (2.54 cm) from the top edge.
  On all pages, the bottom margin should be $1\frac{1}{8}$ inches (2.86 cm) from the bottom edge of the page for $8.5 \times 11$-inch paper;
  for A4 paper, approximately $1\frac{5}{8}$ inches (4.13 cm) from the bottom edge of the
  page.

  \begin{table}
    \centering
    \begin{tabular}{@{}lc@{}}
      \toprule
      Method & Frobnability \\
      \midrule
      Theirs & Frumpy \\
      Yours & Frobbly \\
      Ours & Makes one's heart Frob\\
      \bottomrule
    \end{tabular}
    \caption{Results.   Ours is better.}
    \label{tab:example}
  \end{table}


  %-------------------------------------------------------------------------
  \section{Possible Methodology}
We are planning to work on Jupyter notebook and google collab throughout our project. We will be using Github for collaboration and version controlling of our project. We plan to clean and pre-process the raw dataset by removing redundant images and making sure that the dataset is completely balanced.The three architectures that we will be using are: Mobile-Net v3, Shufflenet, and ResNet. Mobile-Net-v3 is a convulation neural network that is said to perform well for mobile devices. We are using Mobile-Net v3 to reduce the training time of the images significantly. Shuffle Net is yet another architecture that performs well for mobile devices. One added advantage here is that Shuffle Net has limited computation power without compromising the accuracy. ResNet is usually recommended for image classification problems. Te architecture has many layers because of which it can accurately represent features, thus increasing its classification powers. We would we using optimization algorithms such as AdaDelta and Adam. Although both the optimization algorithm are computationally expensive, we believe that they will might significantly improve the learning rate of the model. Apart from using optimization algorithms, we would use n-ply cross validation to obtain reliable results and avoid overfitting. To evaluate how well our model performs, we would be using different evaluation metrics such as precision, recall, accuracy and f1 score.

\pagebreak

  %-------------------------------------------------------------------------
  \section{Gantt Chart}

  

  %-------------------------------------------------------------------------
  \subsection{References}

  List and number all bibliographical references in 9-point Times, single-spaced, at the end of your paper.
  When referenced in the text, enclose the citation number in square brackets, for
  example~\cite{Authors14}.
  Where appropriate, include page numbers and the name(s) of editors of referenced books.
  When you cite multiple papers at once, please make sure that you cite them in numerical order like this \cite{Alpher02,Alpher03,Alpher05,Authors14b,Authors14}.
  If you use the template as advised, this will be taken care of automatically.


  



  %%%%%%%%% REFERENCES
  {\small
  \bibliographystyle{ieee_fullname}
  \bibliography{egbib}
  }
  https://www.mindspore.cn/tutorial/training/en/r1.1/advanced_use/cv_resnet50.html
  https://towardsdatascience.com/5-most-well-known-cnn-architectures-visualized-af76f1f0065e
  

  \end{document}
